\chapter{introduction}

Quantum computing is one of the most promising emerging scientific and technological fields of the past decades. The theoretical computational advantages of quantum computing allow for valuable applications in fields such as cryptography, chemistry and machine learning. However, quantum computing is no straightforward task: the fundamental issue is to perform high-fidelity operations on a coherent, scalable set of qubits %\cite{malley}. Another fundamental objective is to manipulate the quantummechanical properties of your qubits as efficiently as possible, using algorithms. Peter Shor's prime factorization algorithm first showed the potential of quantum algorithms, and this research aims to find the most efficient algorithm for the ground state energy problem in chemistry \cite{shor}.


This problem is highly relevant, as three percent of the world's energy output is spent on making fertilizer %\cite{reiher}. We currently rely on a highly outdated, energy-intensive process requiring very large amounts of natural gas. However, there exists an anaerobic bacteria performing a process serving the exact same function while requiring much less energy, utilizing nitrogen fixation with molecules such as FeMoCo. Traditional chemical analyses have not been able to give us an understanding on the details of this process, and as this molecule is highly complex, simulations using classical supercomputers are out of reach. \textcite{reiher} have shown that quantum computers are, even when taking in account the substantial decoherence and error of current gate computations, able to elucidate chemical processes such as nitrogen fixation in nitrogenase \cite{reiher}. Current research within quantum computing has not yet been able to provide significant insight into nitrogenase, and this study aims to build towards understanding more complex chemical reactions with quantum computers by examining the efficiency of a quantum walk ground-energy algorithm on the water molecule. If the scientific community succeeds in elucidating nitrogenase, new and more efficient processes of creating fertilizer may be developed, significantly reducing the global carbon footprint.
