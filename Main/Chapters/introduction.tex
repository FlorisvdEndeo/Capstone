\chapter{Introduction}

Quantum computing is one of the most promising emerging scientific and technological fields of the past decades \cite{McArdle}. The theoretical computational advantages of quantum computing allow for valuable applications in fields such as cryptography, chemistry and machine learning. However, building useful quantum computers is highly complicated: the fundamental issue is to perform high-fidelity operations on a coherent, scalable set of qubits \cite{malley}. Peter Shor's prime factorization algorithm first showed the potential of quantum algorithms and in the years since, the field of quantum computing has rapidly expanded and can be divided into many, interdisciplinary sub-fields \cite{shor}. Among these fields is quantum computational chemistry, which aims to solve classically intractable chemistry problems. Progress within this field might elucidate unresolved problems of solid-state physics, superconductivity at high temperature and some complex chemical reactions \cite{McArdle}. However, as the physical implementation of quantum computers is still severely limited by issues such as decoherence and quantum noise, quantum algorithms need to be optimized to lower the number of gates needed to perform the desired calculations. While quantum algorithms have a range of capabilities within quantum computational chemistry, we focus on one problem in particular, called the \textit{electronic structure problem}: finding the ground state and ground energy of chemical systems. We study this specific problem because knowledge of the energy eigenstates allows predictions of other fundamental properties of the system, such as reaction rates and location of stable structures \cite[p.~3]{McArdle}. Furthermore, we can relatively easily transfer the algorithms and methods used for the electronic structure problem to other problems such as determining the transition states. \\

\vspace{5mm}

There are several methods to classically solve the \textit{electronic structure problem}, such as density functional theory and the Hartree-Fock method. While these calculations are highly efficient, they struggle with large, highly correlated systems, due to the exponential complexity. \textcite{McArdle} state that there is a knowledge gap of systems of about 100-200 spin-orbitals, as such systems are too large and too correlated to accurately calculate with classical methods. Within this range falls the molecule FeMoCo, having 113 electrons in 152 spin-orbitals. \\ To give some context as to why this molecule is of special interest, it is important to note that three percent of the world's energy output is spent on making fertilizer \cite{reiher}. \todo{setup}  We currently rely on a highly energy-intensive process requiring very large amounts of natural gas. However, there exist anaerobic bacteria performing a process serving the exact same function while requiring much less energy, utilizing nitrogen fixation with molecules such as FeMoCo \cite{McArdle}. Traditional chemical analyses have not been able to provide an understanding on the details of this process, and, as this molecule is highly complex, simulations using classical supercomputers are out of reach. \textcite{reiher} have shown that quantum computers are, even when taking in account the substantial decoherence and error of current gate computations, able to provide new insights on chemical processes such as nitrogen fixation in nitrogenase. However, a resource estimation performed by \textcite{berry} shows that the resources needed to simulate FeMoCo is equivalent to roughly one million qubits running for 2 months. While this result has been an improvement over previous estimations, it is clear that it is still out of reach currently \cite{McArdle}. But if the scientific community succeeds in obtaining a greater understanding of nitrogenase, new and more efficient processes of creating fertilizer may be developed, significantly reducing the global carbon footprint. This thesis aims to contribute towards minimizing gate count by exploring an algorithm which may provide speedup in solving the electronic structure problem.

\vspace{5mm}

The general set of techniques used to simulate a quantum system with a quantum computer is called \textit{Hamiltonian simulation} \cite{cao}. Perhaps the most used Hamiltonian simulation method is the Variational Quantum Eigensolver (VQE). The VQE prepares a trial wave function and approximates the energy by iteratively retrieving the Pauli operator tensor products. This algorithm is often used on hybrid computers, as the optimization algorithms required are much faster on a classical computer than a quantum computer. While VQE is a very practical application for near-term quantum computers, methods using phase estimation, which will be explained in section \ref{sec:pea}, offer more efficient circuit implementations in terms of gate complexity \cite{bian}. The standard approach is to approximate the time-evolution operator of the Hamiltonian. However, this is prone to approximation errors. \textcite{poulin} have recently proposed another algorithm that circumvents the need for approximating the time-evolution operator by \textit{qubitization}, using ideas and techniques from quantum walks.

\vspace{5mm}

In this thesis, we will thoroughly analyze the algorithm proposed by \textcite{poulin} and adapt an implementation of the algorithm for two molecular systems. The research process of this capstone is composed of three main components. The first component (Chapter 2) consists of a literature review of the fundamentals of quantum computing and quantum simulation, introducing standard conventions and treating concepts appearing frequently in later sections. To do so, several established texts such as \textcite{nielsen}, \textcite{McArdle} and \textcite{cao} will be synthesised. The second component (Chapter 3)  expands on the algorithm, as well as giving proofs of the mathematical claims made by \textcite{poulin}. The third component (Chapter 4) proposes a circuit implementation of the algorithm and adapts the ProjectQ \footnote{http://projectq.ch} implementation made by \textcite{steiger} for two Hamiltonians of $H_2$ and $H$. Finally, the results of the implementation and the prospects of this algorithm are discussed.
