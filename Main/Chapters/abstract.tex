\abstract{Transitionary sentence of qchem. A fundamental problem in chemistry is finding the ground state and ground energy of a molecule, called the electronic structure problem. While classical computers struggle with solving large, highly correlated systems due to the exponentially growing complexity, quantum computers are able to efficiently simulate quantum systems, which can lead to an increased understanding of complex chemical reactions. If quantum computers succeed in providing insight in nitrogenase, the world's total energy output may be significantly reduced. But despite rapid advancements in all facets of quantum computing, new theoretical advancements are needed to solve the electronic structure problem for complex systems in the future. \\ We discussed the Spectrum by quantum walk algorithm proposed by \textcite{poulin}, where the main idea is to ``simulate not imitate'' \cite[p.~5]{poulin}. While methods such as Trotterization attempt to approximate the time evolution operator $U(t)$, this algorithm instead employs another, exactly implementable unitary operator $W$ which is a function of the Hamiltonian. In Chapter 3, we have provided proofs of the mathematical properties of the operations that $W$ consists of and we have shown the relation of the spectrum of the Hamiltonian with the spectrum of $W$. Energy measurements can be performed 







}
