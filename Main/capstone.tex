\documentclass{report}

\usepackage{amsmath}
\usepackage[utf8]{inputenc}
\usepackage{mathtools}
\usepackage{tikz}
\usepackage{physics}
\usepackage[backend=biber]{biblatex}
\usepackage{biblatex}
\usepackage{titling}
\usepackage{titletoc}
\usepackage{kbordermatrix}
\usepackage{graphicx}
\usepackage[document]{ragged2e}
\usepackage{fancyhdr}
\usepackage{bbold}
%\usepackage{babylon}
\usepackage{tikz}
\usetikzlibrary{quantikz}
%\usepackage{qcircuit}
\usepackage{blochsphere}
\usepackage{hyperref}
\usepackage{enumitem}
\usepackage{stmaryrd}
\usepackage{subcaption}
\usepackage{import}
\usepackage[colorinlistoftodos]{todonotes}
\usepackage[top=1in, bottom=1in, left=1.25in, right=1.25in]{geometry}
%%% Todonotes commands %%%
\newcommand{\addref}{\todo[color=red!40]{Add reference.}}
\newcommand{\question}{\todo[color=green!40]}


\def\bracket#1#2{\langle #1 | #2 \rangle}
\def\kb#1#2{| #1 \rangle\!\langle #2 |}
\def\kbtwo#1#2#3#4{|#1 \rangle\!\langle #2 | #3 \rangle\!\langle #4 | }



\newcommand{\Sop}{{(I - 2\dyad{\beta}{\beta}) \otimes I}}
\newcommand{\Vop}{{\sum_{j}\kb j j  \otimes P_{j}}}
\newcommand{\fiok}{{\sum_j\beta_j\ket{j }\otimes \ket{\phi_0}}}
\newcommand{\fiik}{{\frac{1}{\sqrt{1-E_0^2}}(V-E_0)\ket{\varphi_0^0}}}
\newcommand{\fiob}{{\sum_m \beta_m^* \bra{m} \otimes \bra{phi_0}}}
\newcommand{\fiib}{{\frac{1}{\sqrt{1-E_0^2}}(V-E_0) \sum_m \beta_m^* \bra{m} \otimes \bra{\phi_0}}}
\newcommand{\fact}{{\frac{1}{1 - E^2_k} (V - E_k)}}
\date{\today}
\newcommand{\facto}{{\frac{1}{\sqrt{1 - E^2_0}} (V- E_0)}}


\newcommand{\lefta}{\bra{\beta}\otimes\bra{\phi_k}}
\newcommand{\leftb}{i\bra{\beta} \otimes \bra{\phi_k} \frac{V-E_k}{\sqrt{1-E_k^2}}}
\newcommand{\righta}{\ket{\beta}\otimes \ket{\phi_k}}
\newcommand{\rightb}{i\frac{V-E_k}{\sqrt{1-E_k^2}}\ket{\beta}\otimes  \ket{\phi_k}}

\newcommand\chap[1]{\input{./Chapters/#1}}

\providecommand\given{}
% can be useful to refer to this outside \Set
\newcommand\SetSymbol[1][]{%
  \nonscript\:#1\vert
  \allowbreak
  \nonscript\:
  \mathopen{}}
\DeclarePairedDelimiterX\Set[1]\{\}{%
  \renewcommand\given{\;\SetSymbol[\delimsize]\;}
  #1
}

\renewcommand\S{\Set*}

\newcount\colveccount
\newcommand*\colvec[1]{
        \global\colveccount#1
        \begin{pmatrix}
        \colvecnext
}
\def\colvecnext#1{
        #1
        \global\advance\colveccount-1
        \ifnum\colveccount>0
                \\
                \expandafter\colvecnext
        \else
                \end{pmatrix}
        \fi
}

\addbibresource{capstone.bib}
%\title{Simulating the Water Molecule using the Spectrum by Quantum Walk Algorithm}
\author{Floris van den Ende}

\begin{document}


\newcommand\email[1]{
    \href{mailto:#1}{\url{#1}}
}

% Takes: name, affiliation & email
\newcommand\authorblock[3]{
    #1\\
    \small{#2}\\
    \small{\email{#3}}
}

\newcommand\keywords[1]{
    \vspace{5mm}\noindent
    \small{\textbf{\textit{Keywords ---}} #1}
}

\title{The Dynamics of the Spectrum by Quantum Walk Algorithm.}



\begin{titlepage}
    \newcommand{\HRule}{\rule{\linewidth}{0.5mm}}

	\center

	\HRule\\[0.4cm]

	{\huge\bfseries \thetitle\\[0.4cm]}

	\HRule\\[1.5cm]

	\begin{minipage}{0.45\textwidth}
		\begin{flushleft}
			\large
			\textit{Author}\\
            Floris van den Ende\\
            {\small Amsterdam University College}\\
            {\small\email{florisvdende@gmail.com}}
		\end{flushleft}
	\end{minipage}
	~
	\begin{minipage}{0.4\textwidth}
		\begin{flushright}
			\large
			\textit{Supervisor}\\
            Dhr. Dr. J.van Wezel\\
            {\small{Faculteit der Natuurwetenschappen en Informatica \& QuSoft}\\
            {\small{\url{j.vanwezel@uva.nl}}}}
		\end{flushright}
    \end{minipage}
    \\[1cm]
    \begin{minipage}[t]{0.45\textwidth}
		\begin{flushleft}
			\large
			\textit{Tutor}\\
            Dr. Forrest Bradbury\\
            {\small Amsterdam University College}\\
            {\small\email{f.bradbury@auc.nl}}
		\end{flushleft}
	\end{minipage}
	~
	\begin{minipage}[t]{0.4\textwidth}
		\begin{flushright}
			\large
			\textit{Daily Supervisor}\\
            Joris Kattemölle\\
            {\small QuSoft}\\
            {\small\email{j.j.kattemolle@uva.nl}}
		\end{flushright}
	\end{minipage}
    \vfill\vfill\vfill

    {\large
        Major: Sciences\\[0.5cm]
%        \thedate

           }
           \vfill\vfill
    \includegraphics[width=0.2\textwidth]{{auc-logo.png}}\\[1cm]

	\vfill


\end{titlepage}


\tableofcontents

\chap{abstract}

\chap{introduction}

\chap{quantumcomp.tex}

\chap{quantumwalk.tex}

\newpage

Meeting Joris 8 May agenda:

\begin{itemize}
  \item Practical details: proposal of Draft deadline (16 May), reader (Christian Majenz), presentation details, Timeline (extension from AUC) -> discuss at the end of Meeting
  \item I now see my interpretation of the aim of this algorithm was wrong. I'd like to walk you through how I see it now.
  Specific questions on about it:
\begin{itemize}
  \item Second quantized Hamiltonian, okay we simplify that. For physical implementation, Jordan Wigner or not?
  \item How does this work with $ \tilde{H} = H_0 + gV$
  \item Hodge Star? Confused about formula.
\end{itemize}
  \item Mathematics: We established that the calculation for $SV$ was nonsense, but I don't see how we can find the values without it
  \item So now scope Capstone: I'll go more in detail of how Zeno effect is manipulated to avoid the need for explicit approximation of time operator.
\end{itemize}



\printbibliography

\end{document}
