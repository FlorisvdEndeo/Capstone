%TC:macro \todo 1
%TC:macro \question 1
%TC:macro \footnote [text]
%TC:envir gather* [] dmath

\documentclass{report}


\usepackage{amsmath}
\usepackage[utf8]{inputenc}
\usepackage{mathtools}
\usepackage{tikz}
\usepackage{physics}
\usepackage[backend=biber]{biblatex}
\usepackage{biblatex}
\usepackage{titling}
\usepackage{titletoc}
\usepackage{kbordermatrix}
\usepackage{graphicx}
\usepackage[document]{ragged2e}
\usepackage{fancyhdr}
\usepackage{bbold}
%\usepackage{babylon}
\usepackage{tikz}
\usetikzlibrary{quantikz}
%\usepackage{qcircuit}
\usepackage{blochsphere}
\usepackage{hyperref}
\usepackage{enumitem}
\usepackage{stmaryrd}
\usepackage{subcaption}
\usepackage{import}
\usepackage[colorinlistoftodos]{todonotes}
\usepackage[top=1in, bottom=1in, left=1.25in, right=1.25in]{geometry}
%%% Todonotes commands %%%
\newcommand{\addref}{\todo[color=red!40]{Add reference.}}
\newcommand{\question}{\todo[color=green!40]}


\def\bracket#1#2{\langle #1 | #2 \rangle}
\def\kb#1#2{| #1 \rangle\!\langle #2 |}
\def\kbtwo#1#2#3#4{|#1 \rangle\!\langle #2 | #3 \rangle\!\langle #4 | }



\newcommand{\Sop}{{(I - 2\dyad{\beta}{\beta}) \otimes I}}
\newcommand{\Vop}{{\sum_{j}\kb j j  \otimes P_{j}}}
\newcommand{\fiok}{{\sum_j\beta_j\ket{j }\otimes \ket{\phi_0}}}
\newcommand{\fiik}{{\frac{1}{\sqrt{1-E_0^2}}(V-E_0)\ket{\varphi_0^0}}}
\newcommand{\fiob}{{\sum_m \beta_m^* \bra{m} \otimes \bra{phi_0}}}
\newcommand{\fiib}{{\frac{1}{\sqrt{1-E_0^2}}(V-E_0) \sum_m \beta_m^* \bra{m} \otimes \bra{\phi_0}}}
\newcommand{\fact}{{\frac{1}{1 - E^2_k} (V - E_k)}}
\date{\today}
\newcommand{\facto}{{\frac{1}{\sqrt{1 - E^2_0}} (V- E_0)}}


\newcommand{\lefta}{\bra{\beta}\otimes\bra{\phi_k}}
\newcommand{\leftb}{i\bra{\beta} \otimes \bra{\phi_k} \frac{V-E_k}{\sqrt{1-E_k^2}}}
\newcommand{\righta}{\ket{\beta}\otimes \ket{\phi_k}}
\newcommand{\rightb}{i\frac{V-E_k}{\sqrt{1-E_k^2}}\ket{\beta}\otimes  \ket{\phi_k}}

\newcommand\chap[1]{\input{./Chapters/#1}}

\providecommand\given{}
% can be useful to refer to this outside \Set
\newcommand\SetSymbol[1][]{%
  \nonscript\:#1\vert
  \allowbreak
  \nonscript\:
  \mathopen{}}
\DeclarePairedDelimiterX\Set[1]\{\}{%
  \renewcommand\given{\;\SetSymbol[\delimsize]\;}
  #1
}

\renewcommand\S{\Set*}

\newcount\colveccount
\newcommand*\colvec[1]{
        \global\colveccount#1
        \begin{pmatrix}
        \colvecnext
}
\def\colvecnext#1{
        #1
        \global\advance\colveccount-1
        \ifnum\colveccount>0
                \\
                \expandafter\colvecnext
        \else
                \end{pmatrix}
        \fi
}

\newcommand\tm[1]{\frac{#1}{\sqrt{1-E_k^2}}}

\addbibresource{capstone.bib}
%\title{Simulating the Water Molecule using the Spectrum by Quantum Walk Algorithm}
\author{Floris van den Ende}

%\usepackage{qcircuit}

\begin{document}  


\newcommand\email[1]{
    \href{mailto:#1}{\url{#1}}
}

% Takes: name, affiliation & email
\newcommand\authorblock[3]{
    #1\\
    \small{#2}\\
    \small{\email{#3}}
}

\newcommand\keywords[1]{
    \vspace{5mm}\noindent
    \small{\textbf{\textit{Keywords ---}} #1}
}

\title{The Dynamics of the Spectrum by Quantum Walk Algorithm.}



\begin{titlepage}
    \newcommand{\HRule}{\rule{\linewidth}{0.5mm}}

	\center

	\HRule\\[0.4cm]

	{\huge\bfseries \thetitle\\[0.4cm]}

	\HRule\\[1.5cm]

	\begin{minipage}{0.45\textwidth}
		\begin{flushleft}
			\large
			\textit{Author}\\
            Floris van den Ende\\
            {\small Amsterdam University College}\\
            {\small\email{florisvdende@gmail.com}}
		\end{flushleft}
	\end{minipage}
	~
	\begin{minipage}{0.4\textwidth}
		\begin{flushright}
			\large
			\textit{Supervisor}\\
            Dhr. Dr. J.van Wezel\\
            {\small{Faculteit der Natuurwetenschappen en Informatica \& QuSoft}\\
            {\small{\url{j.vanwezel@uva.nl}}}}
		\end{flushright}
    \end{minipage}
    \\[1cm]
    \begin{minipage}[t]{0.45\textwidth}
		\begin{flushleft}
			\large
			\textit{Tutor}\\
            Dr. Forrest Bradbury\\
            {\small Amsterdam University College}\\
            {\small\email{f.bradbury@auc.nl}}
		\end{flushleft}
	\end{minipage}
	~
	\begin{minipage}[t]{0.4\textwidth}
		\begin{flushright}
			\large
			\textit{Daily Supervisor}\\
            Joris Kattemölle\\
            {\small QuSoft}\\
            {\small\email{j.j.kattemolle@uva.nl}}
		\end{flushright}
	\end{minipage}
    \vfill\vfill\vfill

    {\large
        Major: Sciences\\[0.5cm]
%        \thedate

           }
           \vfill\vfill
    \includegraphics[width=0.2\textwidth]{{auc-logo.png}}\\[1cm]

	\vfill


\end{titlepage}


\tableofcontents

\listoftodos

\todo{what is k}
\todo{size of matrices and computational complexity}
\question{what is difference between adiabatic and this}
\todo{prove V is hermitian}

\[
    \Qcircuit @C=1em @R=1em @!R {
      & \lstick{\ket{q_0}} & \qw & \targ & \qw & \meter & \qw\\
      & \lstick{\ket{q_1}} & \gate{H} & \ctrl{-1} & \meter & \qw & \qw
    }
  \]

\abstract{Transitionary sentence of qchem. A fundamental problem in chemistry is finding the ground state and ground energy of a molecule, called the electronic structure problem. While classical computers struggle with solving large, highly correlated systems due to the exponentially growing complexity, quantum computers are able to efficiently simulate quantum systems, which can lead to an increased understanding of complex chemical reactions. If quantum computers succeed in providing insight in nitrogenase, the world's total energy output may be significantly reduced. But despite rapid advancements in all facets of quantum computing, new theoretical advancements are needed to solve the electronic structure problem for complex systems in the future. \\ We discussed the Spectrum by quantum walk algorithm proposed by \textcite{poulin}, where the main idea is to ``simulate not imitate'' \cite[p.~5]{poulin}. While methods such as Trotterization attempt to approximate the time evolution operator $U(t)$, this algorithm instead employs another, exactly implementable unitary operator $W$ which is a function of the Hamiltonian. In Chapter 3, we have provided proofs of the mathematical properties of the operations that $W$ consists of and we have shown the relation of the spectrum of the Hamiltonian with the spectrum of $W$. Energy measurements can be performed 







}


\chapter{introduction}

Quantum computing is one of the most promising emerging scientific and technological fields of the past decades. The theoretical computational advantages of quantum computing allow for valuable applications in fields such as cryptography, chemistry and machine learning. However, quantum computing is no straightforward task: the fundamental issue is to perform high-fidelity operations on a coherent, scalable set of qubits \cite{malley}. Another fundamental objective is to manipulate the quantummechanical properties of your qubits as efficiently as possible, using algorithms. Peter Shor's prime factorization algorithm first showed the potential of quantum algorithms, and present research aims to find the most efficient algorithm for the ground state energy problem in chemistry \cite{shor}.


This problem is highly relevant, as three percent of the world's energy output is spent on making fertilizer \cite{reiher}. We currently rely on a highly outdated, energy-intensive process requiring very large amounts of natural gas. However, there exists an anaerobic bacteria performing a process serving the exact same function while requiring much less energy, utilizing nitrogen fixation with molecules such as FeMoCo. Traditional chemical analyses have not been able to give us an understanding on the details of this process, and as this molecule is highly complex, simulations using classical supercomputers are out of reach. \textcite{reiher} have shown that quantum computers are, even when taking in account the substantial decoherence and error of current gate computations, able to elucidate chemical processes such as nitrogen fixation in nitrogenase \cite{reiher}. Current research within quantum computing has not yet been able to provide significant insight into nitrogenase, and this study aims to build towards understanding more complex chemical reactions with quantum computers by examining the efficiency of a quantum walk ground-energy algorithm on the water molecule. If the scientific community succeeds in elucidating nitrogenase, new and more efficient processes of creating fertilizer may be developed, significantly reducing the global carbon footprint.


The most fundamental property of a static molecule is the energy of the  molecule in its ground state. Before any other significant conclusions can be drawn about a molecule, the ground state energy must be known and accurately measured. If the field of quantum computing wishes to contribute to the understanding of chemical processes, it must succeed in efficiently retrieving the eigenvalues and eigenstates of the Hamiltonian operator of the molecule. The water molecule is simple enough for a classical computer to simulate and the ground state energy can be determined with the Hartree-Fock method, allowing this thesis to find the accuracy of the results of the algorithm \cite{smith}.

The second quantized Hamiltonian of the static water molecule generally is of the form \cite{bian}:

$$
H=\sum_{i, j=1}^{12} h_{i j} a_{i}^{\dagger} a_{j}+\frac{1}{2} \sum_{i, j, k, l=1}^{12} h_{i j k l} a_{i}^{\dagger} a_{j}^{\dagger} a_{k} a_{l},
$$

where $ a_{i}^{\dagger}$ and $a_{j}$ are the fermionic creation and annihilation operators and $ h $ is the respective interaction coefficient. In the ground state, we can assume that the two uppermost orbitals of the oxygen atom are unoccupied. In that case, 12 spin-orbitals have to be taken in consideration and the Hamiltonian is a sum over each one of these spin states.
\textcite{bian} reduce this Hamiltonian, using parity transformations and simplifications to the following form:

$$
H=\sum_{i=j}^{N} \alpha_{j} P_{j}
$$

Here, $\alpha_{j}$ is a function of the hydrogen-hydrogen bond length, and $ P_{j}$ is a multi-qubit Pauli operator, a tensor product of the four Pauli operators which are defined as the following \cite{malley}:


$$
X=\left(\begin{array}{cc}
0 & 1 \\
1 & 0
\end{array}\right), \quad \quad Y=\left(\begin{array}{cc}
0 & -i \\
i & 0
\end{array}\right), \quad \quad Z=\left(\begin{array}{cc}
1 & 0 \\
0 & -1
\end{array}\right),\quad \quad I=\left(\begin{array}{cc}
1 & 0 \\
0 & 1
\end{array}\right)
$$




An example of a multi-qubit Pauli operator would then be: $$ P = Z \otimes X.$$
The Hamiltonian consists of a sum of up to $N$ terms, where in this case $N = 2^4 = 16$, equal to the number of possible multi-qubit operators. Once the Hamiltonian is reduced, the ground energy can be derived from the Hamiltonian. There are several well-established algorithms. \textcite{bian} compare five methods for calculating the ground-state energy of the water molecule, namely:

\begin{itemize}
  \item \emph{Trotter Phase Estimation Algorithm}
  \\ This method uses Trotter-Suzuki decompositio(n to approximate  the propagating term $ e^{-i \alpha_{i} h_{i} t}$, and subsequently extracts the ground state energy from the phase. The Trotter PEA requires $\mathcal{O}(n)$ qubits to run and has a gate complexity of $\mathcal{O}(\frac{n^{5}}{(\epsilon / A)^{2}})$, where $n$ is the number of orbital basis functions, $A=\sum_{i=1}^{L}\left|\alpha_{i}\right|$ and $\epsilon$ is the desired accuracy of the energy.
  \item \emph{Direct Implementation of Hamiltonian in First Order}
  \\This method largely relies on the same principles as the Trotter PEA, but it employs a different unitary operator $U$. This Direct-PEA requires $\mathcal{O}(n)$ qubits to run and has a gate complexity of $\mathcal{O}(\frac{n^{5}}{(\epsilon / A)^{2}}).$
  \item \emph{Direct Implementation of Hamiltonian in Second Order}
  \\This method is the same as the First Order Direct-PEA, but it approximates $U$ to the second order instead. This variant of the Direct-PEA requires $\mathcal{O}(n)$ qubits to run and has a gate complexity of $\mathcal{O}(\frac{n^{5}}{(\epsilon / A)^{1.3}}).$
  \item \emph{Direct Measurement of Hamiltonian}
  \\This method translates the Hamiltonian directly into a quantum circuit and retrieves the ground energy by repeatedly measuring. Direct measurement requires $\mathcal{O}(n)$ qubits to run and has a gate complexity of $\mathcal{O}(n^{5}).$
  \item \emph{Variational Quantum Eigensolver}
  \\The VQE prepares a trial wave function and approximates the energy by iteratively retrieving the Pauli operator tensor products. This algorithm is often used on hybrid computers, as the the optimization algorithms required are much faster on a classical computer than a quantum computer. The VQE only requires $n$ qubits to run, and has a gate complexity of $\mathcal{O}(n^{2} d)$, where d is the number of iterations of entangling gates.

\end{itemize}

According to \textcite{bian}, it is evident that all five methods are viable options for solving the ground energy problem. There are however some major variations regarding gate complexity and qubit number. Second order Direct-PEA has the best gate complexity, but the VQE seems to be the most practical method for the near future, as it has decently efficient gate computations and, most importantly, requires the least amount of qubits.

$$\Sop = \Vop$$

The number of possible methods for the ground state problem are obviously not limited to the aforementioned five. \textcite{poulin} recently proposed another method named Spectrum by Quantum Walk, which bypasses the need to approximate the time-evolution operator, by initializing the quantum computer to an invariant subspace of the unitary operator. Doing so allows you to use any unitary operator which is a function of the Hamiltonian. \textcite{poulin} offer a specific unitary operator which, when initialized in the right bases, can be precisely implemented, avoiding any errors that might come with Trotter or Taylor approximations. Currently, no research has applied the Spectrum by Quantum Walk algorithm to the water molecule.
\section{Methodology}


The research process of this capstone will be composed of three main components. The first component consists of a literature review on the fundamentals on quantum computing, introducing standard conventions and treating concepts appearing frequently in later sections. To do so, several established texts such as \textcite{nielsen}, \textcite{steane} and \textcite{Divincenzo} will be synthesised. The second component expands on and provides context for the algorithm proposed by \textcite{poulin}, and it will also elucidate the relevant atomic structure of the water molecule in the form of a literature review. Furthermore, this part will also cover the implementation of the algorithm into a quantum circuit using Cirq \footnote{https://github.com/quantumlib/Cirq}, Google's open-source Python-based quantum computing package. The final part of this thesis will apply the quantum circuit of the algorithm to the specific Hamiltonian of the water molecule, retrieving the eigenstates of the Hamiltonian. Ultimately, this paper aims to develop an order of gate complexity as to compare the efficiency of the quantum walk algorithm to the five methods presented by \textcite{poulin}.





\subsection{Quantummechanical foundations}

Before we delve into the specifics of quantummechanics, a basic understanding of the quantummechanical principles governing qubits is needed. The state of some general quantummechanical system is described by its wavefunction, $\phi$. A time-independent wavefunction is often expressed as some linear combination of orthonormal basis states, such that.


However, as the natural language of quantum mechanics is linear algebra, it is commonpractice to describe states in bra-ket notation (Source). States are described as vectors, and the simplest arbitratry quantummechanical system can be written:
$$\ket{\psi} = \alpha \ket{0} + \beta \ket{1},$$

where $\psi$ has to satisfy the normalization condition, implying that $\alpha^2 + \beta^2 = 1$. This system is called the \textit{qubit}, which lives in a two-dimensional Hilbert space. The defining feature of qubits is that they are in quantum superposition. When measuring a qubit, the outcome of the measurement in the so-called computational basis will still be either $\ket{0} \text{or} \ket{1}$, but these outcomes occur with probabilities depending on the superposition of the state, namely $\abs{\alpha} \text{or} \abs{\beta}$. Applying quantum logic gates manipulates both basis states at the same time, resulting in an increased computational potential when compared to classical bits. The orthonormal basis states $\ket{0} \text{and} \ket{1}$ are generally just an abstract representation of a two-level system, but depending on the type of quantum computer, they can be represented physical properties such as spin in quantum dots as well as energy levels in trapped ions \cite{mcardle}.

Whenever there is a system with $n$ qubits, the joint state of that system is described by taking the tensor product of each individual state of that system. Taking a tensor product of two states is mathematically equivalent to taking the Kronecker product of the vectors of that state. For some arbitrary states $\ket{\phi} = \colvec{2}{\alpha}{\beta}$ and $\ket{\psi} = \colvec{2}{\gamma}{\delta}$, the joint state is given by

$$
\ket{\phi} \otimes \ket{\psi}= \colvec{2}{\alpha}{\beta} \otimes \colvec{2}{\gamma}{\delta} = \colvec{4}{\alpha\gamma}{\alpha\delta}{\beta\gamma}{\beta\delta}
$$

FOUR POSTULATES, SEE SHOR

%\subsection{qauntum circuits}

\subsection{Simulation}

The core objective of simulation is solving the set of differential equations that describe the physical evolution in time of that system. In the case of quantum simulation, the dynamical behaviour of the system is described by the Schrödinger equation,

$$ i \hbar \frac{d}{dt} \ket{\psi} = H \ket{\psi}.$$

Quantum simulation is ``DIFFICULT'' as the number of differential equation grows exponentially with the number of qubits that are evolving, as one qubit requires two differential equations to b solved.


Quantum simulation can be divided into two different sets of problems, the dynamic problem and the static problem. The static problem consists of preparing the (correct) initial state and the dynamical problem consists of reproducing how the time-evolution unitary operator, $U= \exp{-iHt}$ affects the state of the system.


The initial idea that motivated the entire field of quantum simulation, is that the storage needed to describe the wavefunction of a system of $N$ particles grows with $\mathcal{O}(exp(N))$ on a classical computer, while this relation is $\mathcal{O}(N)$ for qubits \cite{cao}. However, one cannot simply retrieve obtain a useful description of the wavefunction stored in qubits, as a classical representation of the wavefunction requires repeated measurements which, due to the collapse of the state during measurements, is a futile process. Instead, alternative processes, such as finding expectation values of observables, have to be utilized. Another important condition that has been satisfied, is that evolving the wavefunction in time is efficiently possible for Hamiltonians that are to a large extent physically realizable \cite{cao}. That is to say that the number of gates needed to simulate a system grows polynomially, rather than exponentially with parameters such as system size and time.\\

 Furthermore, it is important to note that essentially all algorithms within quantum chemistry assume error-free quantum computations. At this point in time, fully coherent quantum computation has not yet been realized, as in the physical realizations of quantum computers today are still too error prone. The process of minimizing decoherence in quantum computers is highly instrumental to quantum computing and is an entire own subfield of quantum computing (Quantum Error Correction), which is out of scope for present research. We refer interested readers to the book Quantum Computation and Quantum Information by \textcite{nielsen} for more information.\todo{is this out of place?} The fact that quantum chemistry algorithms don't account for errors mean that none of these algorithms are executable in a non-trivial way yet i.e. deliver results that classical computers couldn't have obtained. This doesn't mean that these algorithms cannot be implemented, as we will see a rudimentary implementation of a specific quantum simulation algorithm later.\\




Hamiltonian simulation roughly correlates to the time-dependent part of simulation as outlined before \todo{make nice transition}
Different models of systems place different constraints on the manifestation of the form of the Hamiltonian. For most physical systems of $n$ particles the Hamiltonian can be written as
$H = \sum_k^L H_k$ where each $H_k$ acts on a specific subsystem. Often these terms are two-body interaction terms such as $X_i X_j$ or even one-body Hamiltonians $X_i$. More specifically, the spin Ising model $
H_{\mathrm{Ising}}=g \sum_{j} X_{j}+J \sum_{\langle i, j\rangle} Z_{i} Z_j
$ consists entirely of a linear combination of such interaction terms. The conventional method for simulating Hamiltonians of such form is to approximate the time-evolution operator such that  $$U(t) \approx\left(e^{-i H_{1} t / n} \ldots e^{-i H_{\ell} t / n}\right)^{n}$$. The method for approximating $U$ as such is called \textit{Trotter decomposition}, and as the gatecount grows polynomially with the number of qubits and time, this method is efficient for
local-interaction Hamiltonians \cite{cao}. \\

However, not all Hamiltonians satisfy the local-interaction condition of the Trotter decomposition, such as the electronic structure problem. The Hamiltonian of the electronic structure problem of $K$ nuclei and $N$ electrons is the following \cite{McArdle}:

$$
\begin{aligned}
H=&-\sum_{i} \frac{\hbar^{2}}{2 m_{e}} \nabla_{i}^{2}-\sum_{I} \frac{\hbar^{2}}{2 M_{I}} \nabla_{I}^{2}-\sum_{i, I} \frac{e^{2}}{4 \pi \epsilon_{0}} \frac{Z_{I}}{\left|\mathbf{r}_{i}-\mathbf{R}_{I}\right|} \\
&+\frac{1}{2} \sum_{i \neq j} \frac{e^{2}}{4 \pi \epsilon_{0}} \frac{1}{\left|\mathbf{r}_{i}-\mathbf{r}_{j}\right|}+\frac{1}{2} \sum_{I \neq J} \frac{e^{2}}{4 \pi \epsilon_{0}} \frac{Z_{I} Z_{J}}{\left|\mathbf{R}_{I}-\mathbf{R}_{J}\right|}
\end{aligned}.
$$

Here, $M_I , \mathbf{R}_{I} ,$  and $Z_I$ repreent respectively the mass, the position, and the atomic number of the Ith nucleus. $\mathbf{r}_{i}$ represents the positon of the $i$-th electron. Since a nuclei is many times heavier than an electron we can utilise the Born-Oppenheimer approximation,  neglecting the electron's position and treating the nuclei as classical point charges \cite{McArdle}. The approximated Hamiltonian becomes:

$$
H_{e}=-\sum_{i} \frac{\nabla_{i}^{2}}{2}-\sum_{i, I} \frac{Z_{I}}{\left|\mathbf{r}_{i}-\mathbf{R}_{I}\right|}+\frac{1}{2} \sum_{i \neq j} \frac{1}{\left|\mathbf{r}_{i}-\mathbf{r}_{j}\right|}
$$

This Hamiltonian however is in it's current form still to little use from a quantum computing perspective, as we cannot translate this to qubits. Therefore we now write this Hamiltonian in the second quantization formalism of Quantum Field Theory. An in-depth explanation of the quantisation formalism is out of scope for this research, but the rough idea of second quantisation is that \textit{fields} are being quantised rather than variables. The second quantised form of the electronic Hamiltonian corresponds to the following.

\[
H=\sum_{p, q} h_{p q} a_{p}^{\dagger} a_{q}+\frac{1}{2} \sum_{p, q, r, s} h_{p q r s} a_{p}^{\dagger} a_{q}^{\dagger} a_{r} a_{s}
\]
where
\[
\begin{aligned}
h_{p q} &=\int \mathrm{d} \mathbf{x} \phi_{p}^{*}(\mathbf{x})\left(-\frac{\nabla^{2}}{2}-\sum_{I} \frac{Z_{I}}{\left|\mathbf{r}-\mathbf{R}_{I}\right|}\right) \phi_{q}(\mathbf{x}) \\
h_{p q r s} &=\int \mathrm{d} \mathbf{x}_{1} \mathrm{d} \mathbf{x}_{2} \frac{\phi_{p}^{*}\left(\mathbf{x}_{1}\right) \phi_{q}^{*}\left(\mathbf{x}_{2}\right) \phi_{r}\left(\mathbf{x}_{2}\right) \phi_{s}\left(\mathbf{x}_{1}\right)}{\left|\mathbf{r}_{1}-\mathbf{r}_{2}\right|.}
\end{aligned}
\]

$a_{p}^{\dagger}$ and $ a_{q}$ are respectively the fermionic creation and annihalation operators. These operators excite or de-excite electrons into spin-orbitals, and obey the following anti-commutation relations
$$
\begin{array}{l}
\left\{a_{p}, a_{q}^{\dagger}\right\}=a_{p} a_{q}^{\dagger}+a_{q}^{\dagger} a_{p}=\delta_{p q} \\
\left\{a_{p}, a_{q}\right\}=\left\{a_{p}^{\dagger}, a_{q}^{\dagger}\right\}=0
\end{array}.
$$

To simulate the electronic structure problem in its second quantized form, a map from operators acting on indistinguishable fermions to operators acting on distinguishable qubits is needed \cite{McArdle}. There are several methods of implementing this map, such as the Jordan Wigner transformation and the Bravyi-Kitaev encoding.
\\

The Jordan-Wigner transformation writes the annihalation and creation operators in terms of Pauli matrices, such that \cite{daskin}

$$
a_{j} \rightarrow I^{\otimes n-j-1} \otimes \sigma_{+} \otimes \sigma_{z}^{\otimes j}, \text { and } a_{j}^{\dagger} \rightarrow I^{\otimes n-j-1} \otimes \sigma_{-} \otimes \sigma_{z}^{\otimes j}
$$

$$
\sigma_+ = \begin{pmatrix} 0&0\\1&0 \end{pmatrix} \
\sigma_- = \begin{pmatrix} 0&1\\0&0 \end{pmatrix}
$$


The Bravyi-Kitaev expression for the operators in terms of Pauli matrices is significantly more involved than the respective Jordan-Wigner expression, and can be found in \cite{seeley}. Applying any of these two encodings to the second quantised Hamiltonian of the $H_2$ molecule will result in the following representation:
$$
\begin{aligned}
\hat{H}_{B K}=&-0.812611+0.171201 \sigma_{0}^{z}+0.16862325 \sigma_{1}^{z}-0.2227965 \sigma_{2}^{z}+0.171201 \sigma_{1}^{z} \sigma_{0}^{z} \\
&+0.12054625 \sigma_{2}^{z} \sigma_{0}^{z}+0.17434925 \sigma_{3}^{z} \sigma_{1}^{z}+0.04532175 \sigma_{2}^{x} \sigma_{1}^{z} \sigma_{0}^{x}+0.04532175 \sigma_{2}^{y} \sigma_{1}^{z} \sigma_{0}^{y} \\
&+0.165868 \sigma_{2}^{z} \sigma_{1}^{z} \sigma_{0}^{z}+0.12054625 \sigma_{3}^{z} \sigma_{2}^{z} \sigma_{0}^{z}-0.2227965 \sigma_{3}^{z} \sigma_{2}^{z} \sigma_{1}^{z} \\
&+0.04532175 \sigma_{3}^{z} \sigma_{2}^{x} \sigma_{1}^{z} \sigma_{0}^{x}+0.04532175 \sigma_{3}^{z} \sigma_{2}^{y} \sigma_{1}^{z} \sigma_{0}^{y}+0.165868 \sigma_{3}^{z} \sigma_{2}^{z} \sigma_{1}^{z} \sigma_{0}^{z}
\end{aligned}
$$

This Hamiltonian is well suited to quantum computers as it will amount to one $8 \times 8$ matrix acting on a four-qubit state.

\subsection{State preparation and the quantum Zeno effect}


Any quantum algorithm for ground state energy estimation, such as the one discussed in this research, requires (eigen)state preparation, as $H_k \ket{\psi} = E_k \ket{\psi}$ only when $H$ is acting on an eigenstate. In the Spectrum by Quantum Walk algorithm, phase estimation performed as a projective energy measurement \cite{Poulin}. Then it follows that for some approximation of the eigenstate $\ket{\tilde{\phi}}$, the state of the system collapses to the actual ground state $\ket{\phi}$ with probability $\abs{\braket{\tilde{\phi}}{\phi}}^2$. Depending on the form of the Hamiltonian, obtaining $\ket{\tilde{\phi}}$ can be a difficult task. \textcite{poulin} offer a method for obtaining this approximation, which is rather similar to the adiabatic state preparation.

The state preparation sequence of \texcite{poulin} goes as follows. We define the Hamiltonian to be a composite sum of a simpler Hamiltonian $H_0$ and some other interaction term $V$. This simpler Hamiltonian $H_0$ is one we know the ground state of. When considering the electronic structure problem of $H_2$, an example for $H_0$ could be the transverse field Ising model. We can then rewrite the Hamiltonian as a function of some parameter $g$, where $V$ is liinearly related to $g$:

$$
H= H_0 + gV
$$

We are now able to prepare the physical register of our quantum computer in the simpler ground state $ H = \ket{\phi(g=0)}$.  Then we perform a sequence of energy measurements where we increase $g$ slightly for each measurement. Ultimately, if the increments of $g$ are sufficiently small, performing the energy measurement of $H(g=1)$ will result in the wave function collapsing to $ \ket{\phi(g=1)}$, which is the actual ground state of the system we are simulating. \todo{finish section}



\subsection{Nomenclature and important identities}

In this chapter, the Spectrum by Quantum Walk algorithm, proposed by \textcite{poulin}, is discussed. Before this algorithm is examined in further detail, the following small section is dedicated to remove any confusion about conventions or identities used later on. \\
Within quantum computing, $j$ represents a binary bit string. In present thesis, a four-qubit system is considered, so for consistency's sake all bit strings are expressed in the four-bit basis.
For example, the bit-string of eleven is represented in vector form by the following:
$$ \ket{11} = \ket{1011} = \ket{1} \otimes \ket{0} \otimes \ket{1} \otimes \ket{1} = \colvec{2}{0}{1} \otimes \colvec{2}{1}{0} \otimes \colvec{2}{0}{1} \otimes \colvec{2}{0}{1} =   \colvec{16}{0}{0}{0}{0}{0}{0}{0}{0}{0}{0}{0}{1}{0}{0}{0}{0}.$$

Or generally, some state $\ket{n}$ in $\mathbb{R}^{16}$  can be represented by  \\ $\Set{e_{n+1} \given 1 \leq n + 1 \leq 16}$, where $e_{n+1}$ is a vector with a 1 inserted at the $n+1^{th}$ position, and 0's elsewhere. \\

All bit string states are orthogonal, implying $$\bracket i j = \delta_{i,j} = \begin{cases}
1& \text{if } i = j\\
    0              & \text{otherwise}
\end{cases}.
$$

As for the outer product,

$$\kb i j = \delta_{i,j} \ e_{i+1, j+1}$$
\\

When summing bit string outer products over the $n$-dimensional space, each basis vector in $\mathbb{R}^{n}$ has a contribution of 1,
resulting in an $n$ by $n$ matrix with only entries of 1 on the diagonal, which is the identity matrix:

$$ \sum_{j}^{n} \kb jj = I_n
$$

The expectation value of some observable $\mathbb{A}$ measured in some basis $\{\varphi, \phi\}$ is:
$$
\bra{\varphi} \mathbb{A} \ket{\phi} = \braket{\varphi}{\mathbb{A}\phi} = \langle \mathbb{A^{\dagger}}\varphi|\phi \rangle
$$

However, if $\mathbb{c}$ is some scalar, then:
$$
\bra{\varphi} \mathbb{c} \ket{\phi} = \braket{\varphi}{\mathbb{c}\phi} = \mathbb{c} \braket{\varphi}{\phi}
$$

An important tensor product identity is the following:
$$
(U \otimes V)(|\psi\rangle \otimes|\phi\rangle)=(U|\psi\rangle) \otimes(V|\phi\rangle)
$$

\subsection{Postulate of Operators}
The goal of this algorithm, and of many other quantum simulation algorithms, is to obtain expectation values of the eigenstates of the Hamiltonian of the system. However, while many other algorithms aim to reproduce the dynamics of a quantum system by approxing the time evolution operator, this algorithm aims to find a useful initial state, which is often the ground state of the Hamiltonian. As the Hamiltonian is absolutely fundamental to this method, let us define this first. First we have to assume that the Hamiltonian can be written in the form

$$
H = \sum_{j=0}^N \alpha_j P_j,
$$
where $j$ represents some possible state of the sytem. In the case of a simple molecule, this state would align with the occupancy of certain (spin-)orbitals. We sum these states up to $N$, which is the number of possible states (in the case of a molecule, the number of of possible orbital combinations). $P_j$ here is a multi-qubit Pauli operator, a tensor product of the four Pauli operators which are defined as the following \cite{nielsen}:

\begin{align*}
X&=\left(\begin{array}{cc}
0 & 1 \\
1 & 0
\end{array}\right), \quad \quad Y=\left(\begin{array}{cc}
0 & -i \\
i & 0
\end{array}\right),\\
 Z&=\left(\begin{array}{cc}
1 & 0 \\
0 & -1
\end{array}\right),\quad \quad I=\left(\begin{array}{cc}
1 & 0 \\
0 & 1
\end{array}\right)
\end{align*}

Before we move forward, we must ensure that the Hamiltonian we are working with has mathematically desirable properties, such as normalization. We do so by introducing a scaling factor of $\mathcal{N} = \sum_{j=0}^N |\alpha_j|$ \cite{poulin}.

Then our new, rescaled Hamiltonian which we will call $\bar{H}$, takes the following form:

\begin{equation}
	\bar H = \frac H{\mathcal{N}} = \sum_j|\beta_j|^2 P_j,
\end{equation}

where $\beta_j = \sqrt{|\alpha_j|/\mathcal{N}}$. Due to the nature of the scaling factor $\mathcal{N}$, $\sum_j \abs{\beta_j}^2 = 1.$
\\\textcite{poulin} define $\beta, B, S$ and  $V$ as following:

$$
\ket\beta = B\ket 0 = \sum_j \beta_j \ket j$$
$$
S = (B (I - 2 \ket{0}\bra{0}) B^{\dagger}) \otimes I= (I-2\kb \beta \beta) \otimes I)$$
\\
$$
V=\sum_{j}\ket{j}\bra{j} \otimes P_{j}
$$

It will be useful to check if $S$ and $V$ are unitary, so for $S$ we multiply it with it's complex conjugate, which, since $S$ is not complex, is the same as taking the square:

\begin{align*}
SS^* = S^2 &= ((I-2\kb \beta \beta) \otimes I)((I-2\kb \beta \beta) \otimes I)\\
&= (I-2\kb \beta \beta)(I-2\kb \beta \beta) \otimes I^2 \\
&= I^2 - 2I\kb \beta \beta\footnotemark - 2\kb \beta \beta I + 4 \ket{\beta} \bracket \beta \beta \bra \beta\\
&= I^2 - 4\kb \beta \beta + 4\kb \beta \beta = I^2 = I
\end{align*}
\footnotetext{The outer product of a vector only produces elements on the diagonal, so multiplying with identity returns that same outer product}



Same process for the operator $ $V :

\begin{align*}
VV^* = V^2 &= \sum_{j} \kb jj \otimes P_{j} \cdot \sum_{k} \kb kk \otimes P_{k} \\
&= \sum_j \sum_k \kbtwo j j k k \otimes P_j P_k\\
&= \sum_j \sum_k \delta_{j,k} | j \rangle \langle k | \otimes P_j P_k\\
&= \sum_j | j \rangle \langle j | \otimes P_j P_j\\
\end{align*}
\begin{gather*}
\text{As}  \sum_{j}^{n} \kb jj = I_n, \\
V^2 = I
\end{gather*}

\subsection{Orthonormal bases}
This algorithm relies on one fundamental property, namely that ``there exists an invariant subspace of $W$ on which the spectrum of $W$ is the simple function of $H$'' (The operator $W$ will be defined in a later stage). This statement implies that there is such $W$, where because it is a function of the Hamiltonian in a certain subspace, we can retrieve properties of that Hamiltonian from the operator $W$. \textcite{poulin} postulate the following orthonormal basis, in which S and V preserve the substate spanned by those basis, which will be the same substate that $W$ will be initialized to:

\begin{align}
\ket{\varphi_k^0} &= \sum_j\beta_j\ket j\otimes\ket{\phi_k} \\
\ket{\varphi_k^1} &= \frac 1{\sqrt{1-E_k^2}}(V-E_k)\ket{\varphi_k^0},
\end{align}

Here, $k$ is a measure of the state of the eigenfunctions of the hamiltonian. When $k=0$, we are working with the ground-state. the place on the  Let us prove that $\varphi_k^0$ and $\varphi_k^1$ are normalized and orthogonal:

\begin{align*}
\braket{\varphi_k^0}{\varphi_k^0}  &= (\bra{\beta} \otimes \bra{\phi_k})(\ket{\beta} \otimes \ket{\phi_k})\\
&= \braket{\beta}{\beta} \otimes \braket{\phi_k}{\phi_k}\\
&= 1
\end{align*}

\begin{align*}
\braket{\varphi_k^1}{\varphi_k^1}  &= \bra{\varphi_k^0} \frac{1}{\sqrt{1 - E^2_k}}(V-E_k) \frac{1}{\sqrt{1 - E^2_k}}(V-E_k) \ket{\varphi_k^0}\\
&= \bra{\varphi_k^0} \frac{1}{1 - E^2_k} (V-E_k)^2  \ket{\varphi_k^0}\\
&= \frac{1}{1 - E^2_k} \bra{\varphi_k^0} (V^2 - 2 E_kV + E_k^2) \ket{\varphi_k^0}\\
&= \frac{1}{1 - E^2_k} \bra{\varphi_k^0} (I - 2 E_kV + E_k^2) \ket{\varphi_k^0}\\
\end{align*}

As the following identity will appear often, a seperate definition is made .

\begin{align*}
\bra{\varphi_k^0} V \ket{\varphi_k^0} &= \bra{\varphi_k^0} (\sum_j \kb j j \otimes P_j) \sum_l \beta_l \ket{l} \otimes \ket{\phi_k^0}\\
&= \bra{\varphi_k^0}\sum_j \sum_l \beta_l \ket{j} \braket{j}{l} \otimes P_j \ket{\phi_k^0}\\
&=(\sum_m \beta_m^* \bra{m} \otimes \bra{\phi_k}) \sum_j \sum_l \beta_l \ket{j} \delta_{j,l} \otimes P_j \ket{\phi_k}\\
&= \sum_m \sum_j \beta_m^* \beta_j \braket{m}{j} \otimes \braket{\phi_k}{P_j \phi_k}\\
&= \sum_m \sum_j \beta_m^* \beta_j \delta_{m,j} \otimes \braket{\phi_k}{P_j \phi_k}\\
&= \sum_j \abs{\beta_j}^2 \braket{\phi_k}{P_j \phi_k}\\
&= \sum_j \braket{\phi_k}{ \abs{\beta_j}^2  P_j \phi_k}
\end{align*}

As $\bar{H}\ket{\phi_k} = E_k\ket{\phi_k}$ and $\bar{H} = \sum_j \abs{\beta_j}^2  P_j,$
\begin{equation}
\bra{\varphi_k^0} V \ket{\varphi_k^0} = \braket{\phi_k}{E_k \phi_k}
= E_k
\end{equation}

Continuing proof of normality of $\varphi_k^1$ using equation 4:



\begin{align*}
\braket{\varphi_k^1}{\varphi_k^1} &= \frac{1}{1 - E^2_k} \bra{\varphi_k^0} (I - 2 E_kV + E_k^2) \ket{\varphi_k^0}\\
&= \frac{1}{1 - E^2_k} \bra{\varphi_k^0} (I - 2 E_k^2 + E_k^2) \ket{\varphi_k^0}\\
&= \frac{1}{1 - E^2_k} \bra{\varphi_k^0} (I -  E_k^2) \ket{\varphi_k^0}\\
&= \frac{1- E^2_k}{1 - E^2_k} \braket{\varphi_k^0}{\varphi_k^0}\\
&= 1
\end{align*}

Proof of orthogonality, using equation 4
\begin{align*}
\braket{\varphi_k^0}{\varphi_k^1} &= \bra{\varphi_k^0}  \frac{1}{\sqrt{1 - E^2_k}}(V-E_k) \ket{\varphi_k^0}\\
&= \bra{\varphi_k^0}  \frac{1}{\sqrt{1 - E^2_k}}(E_k-E_k) \ket{\varphi_k^0}\\
&= 0
\end{align*}

\subsection{Blockdiagonality}

Poulin et al \cite{poulin} state that the unitary property of $S$ and $V$ allow them to be put in block diagonal form with $ 2 \times 2$ blocks. A general block diagonal matrix only has nonzero entries in the blocks of size nxn on the diagonal, which for $ n = 2$ looks like the following:

\[
\left(\begin{array}{@{}c|c@{}}
  \begin{matrix}
  a & b \\
  c & d
  \end{matrix}
  & \begin{matrix}
  0 & 0 \\
  0 & 0
  \end{matrix} \\
\hline
  \begin{matrix}
    0 & 0 \\
    0 & 0
    \end{matrix} &
  \begin{matrix}
  a & b \\
  c & d
  \end{matrix}
\end{array}\right)
\]

We can put the expectation values of S and V in one of such matrices in the subspace of $\varphi_k^0$ and $\varphi_k^1$. The matrix with placeholder values then looks like the following:

\begin{equation}
\kbordermatrix{&
\ket{\varphi_0^0}&\ket{\varphi_0^1}&\vrule&\ket{\varphi_1^0}&\ket{\varphi_1^1}&\vrule&\cdots&\vrule&\ket{\varphi_k^0}&\ket{\varphi_k^1}\\
\bra{\varphi_0^0}&V_{11}&V_{21}&\vrule&0&0&\vrule&&\vrule&0&0\\
\bra{\varphi_0^1}&V_{12}&V_{22}&\vrule&0&0&\vrule&&\vrule&0&0\\\hline
\bra{\varphi_1^0}&0&0&\vrule&V_{33}&V_{43}&\vrule&&\vrule&0&0\\
\bra{\varphi_1^1}&0&0&\vrule&V_{43}&V_{44}&\vrule&&\vrule&0&0\\\hline
\vdots&&&\vrule&&&\vrule&\ddots&\vrule\\\hline
\bra{\varphi_k^0}&0&0&\vrule&0&0&\vrule&&\vrule&a&b\\
\bra{\varphi_k^1}&0&0&\vrule&0&0&\vrule&&\vrule&c&d\\
}
\end{equation}

The off-diagonal elements are zero as $ \braket{\phi_k}{\phi_l} = \delta_{k,l}$. In other words, all the eigenstates of the Hamiltonian are orthogonal. We can find the values of $S$ and  $V$ in the basis by

\begin{align*}
\bra{\varphi_0^0}S\ket{\varphi_0^0} &= \bra{\varphi_0^0} (I-2\kb \beta \beta) \otimes I \ket{\varphi_0^0}\\
&= \bra{\varphi^0_0} ((I-2\kb \beta \beta) \otimes I) \sum_j\beta_j\ket{j }\otimes \ket{\phi_0} \\
&= \bra{\varphi^0_0} \sum_j (I-2\kb \beta \beta) \beta_j\ket{j} \otimes \ket{\phi_0}\\
&= \bra{\varphi^0_0} - \ket{\beta} \otimes \ket{\phi_0}\\
&= - (\bra{\beta} \otimes \bra{\phi_0})  \ket{\beta} \otimes \ket{\phi_0}\\
&= -1
\end{align*}



Before we find the value of $\bra{\varphi_0^1}S\ket{\varphi_0^1}$, let us find what applying $S$ and $V$ in succession on a general state results:

\begin{align*}
SV&= (\Sop) \Vop\\
&= ((I - 2\kb \beta \beta)\sum_j \kb j j) \otimes I P_j\\
&= \sum_j ((I - 2 \abs{\beta_j}^2 \kb j j) \kb j j) \otimes P_j\\
&= \sum_j (\kb j j - 2 \kbtwo j j j j)\otimes P_j\\
&= - \sum_j \kb jj \otimes P_j\\
&= - V
\end{align*}

As $S$ is a reflexion operator, the result makes sense, since $V$ also preserves the subspace spanned by the orthornomal bases. Let us finish filling out the block diagonal matrix of S, by finding the remaining values.

\begin{align*}
\bra{\varphi_0^1}S\ket{\varphi_0^1} &= \bra{\varphi_0^1} (\Sop) \fiik \\
&= \bra{\varphi_0^1} \frac{1}{\sqrt{1 - E^2_0}} (-V + E_0) (-\ket{\varphi_0^0})\\
&= \bra{\varphi_0^0} \frac{1}{\sqrt{1 - E^2_0}} (V - E_0) \frac{1}{\sqrt{1 - E^2_0}} (-V + E_0) -(\ket{\varphi_0^0})\\
&= - \bra{\varphi_0^0} \frac{1}{1 - E^2_0} (- V^2 +E_0V+E_0V- E_0^2) \ket{\varphi_0^0}\\
\text{As } \bra{\varphi_k^0} V \ket{\varphi_k^0} &= E_k,\\
\bra{\varphi_0^1}S\ket{\varphi_0^1} &= - \bra{\varphi_0^0} \frac{1}{1 - E^2_0} (-1+E_0^2) \ket{\varphi_0^0}\\
&= - \bra{\varphi_0^0} \frac{-(1-E_0^2)}{1 - E^2_0} \ket{\varphi_0^0}\\
&= - \bra{\varphi_0^0} (- \ket{\varphi_0^0})\\
&= \braket{\varphi_0^0}{\varphi_0^0}\\
&= 1\\
\end{align*}


\begin{align*}
\bra{\varphi_0^0}S\ket{\varphi_0^1} &= \bra{\varphi_0^0} (\Sop) \fiik\\
&= \bra{\varphi_0^0} \frac{1}{\sqrt{1 - E^2_0}} (-V + E_0) \ket{-\varphi_0^0}\\
&= - \bra{\varphi_0^1} \frac{1}{\sqrt{1 - E^2_0}} (-E_0 + E_0) \ket{\varphi_0^0}\\
&= - \bra{\varphi_0^1} 0 \ket{\varphi_0^0}\\
&= 0
\end{align*}

As the operator S is symmetric around the diagonal, S is self adjoint and from the identity $$\bra{\psi} A^{\dagger}\ket{\phi}= \bra{\phi} A \ket{\psi}^{*}$$
we can conclude that $\bra{\varphi_0^0}S\ket{\varphi_0^1} = \bra{\varphi_0^1}S\ket{\varphi_0^0} = 0.$


The block of S is now completed, and looks like the following:

\begin{equation}
S = \
\kbordermatrix{& \ket{\varphi_0^0}&\ket{\varphi_0^1}\\
\bra{\varphi_0^0}&-1&0\\
\bra{\varphi_0^1}&0&1}
\end{equation}


From these values, it is clear that there is no $k$-dependency and therefore above matrix can be generalised for all $k$-values. The entirety of matrix 4 can be condensed into the following:


\begin{equation}
S = \kbordermatrix{& \ket{\varphi_k^0}&\ket{\varphi_k^1}\\
\bra{\varphi_k^0}& -1&0\\
\bra{\varphi_k^1}&0&1}
\end{equation}


Let us find the equivalent matrix for the operator $V$. In equation 4, it has been shown that $\bra{\varphi_k^0} V \ket{\varphi_k^0} = E_k$. Moving on,

\begin{align*}
\bra{\varphi_0^0}V\ket{\varphi_0^1}&= \bra{\varphi_0^0}  \frac{1}{\sqrt{1 - E^2_0}} (V^2 - E_0V) \ket{\varphi_0^0}\\
&= \bra{\varphi_0^0} \frac{1}{\sqrt{1 - E^2_0}} (1 - E_^2) \ket{\varphi_0^0}\\
&= \bra{\varphi_0^0} \sqrt{1 - E^2_0} \ket{\varphi_0^0}\\
&= \sqrt{1 - E^2_0} \braket{\varphi_0^0}{\varphi_0^0}\\
&= \sqrt{1 - E^2_0}
\end{align*}

As V is also diagonal and therefore self-adjoint, $$\bra{\psi} A^{\dagger}\ket{\phi}= \bra{\phi} A \ket{\psi}^{*}$$ holds again and thus $$\bra{\varphi_0^0}V\ket{\varphi_0^1} = \bra{\varphi_0^1}V\ket{\varphi_0^0} = \sqrt{1 - E^2_0}$$

Lastly, we need to find $\bra{\varphi_0^1}V\ket{\varphi_0^1}$:

\begin{align*}
\bra{\varphi_0^1}V\ket{\varphi_0^1} &= \bra{\varphi_0^1}  \frac{1}{\sqrt{1 - E^2_0}} (V^2 - E_0V) \ket{\varphi_0^0}\\
&= \bra{\varphi_0^0} \frac{1}{1 - E^2_0} (V - E_0)(1 - E_0V)\ket{\varphi_0^0}\\
&= \bra{\varphi_0^0} \frac{1}{1 - E^2_0} (V - E_0 - E_0 + E_0^2V)\ket{\varphi_0^0}\\
&= \bra{\varphi_0^0} \frac{1}{1 - E^2_0} (-E_0 + E_0^3)\ket{\varphi_0^0}\\
&= \bra{\varphi_0^0} (- E_0 \frac{1 - E^2_0}{1 - E^2_0}\ket{\varphi_0^0})\\
&= \bra{\varphi_0^0} (- E_0 \ket{\varphi_0^0})\\
&= - E_0
\end{align*}

Once again, we can compile these value in the block matrix and generalize to any $k$-value in the range of energy states to obtain:

\begin{equation}
S = \
\kbordermatrix{& \ket{\varphi_k^0}&\ket{\varphi_k^1}\\
\bra{\varphi_k^0}& E_k&\sqrt{1 - E^2_0}\\
\bra{\varphi_k^1}&\sqrt{1 - E^2_0}&-E_k
}
\end{equation}

\subsection{Creating the Unitary Walk Operator}

Before we start thinking about constructing an operator from $S$ and $V$, it is paramount to have a clear picture in mind of how $S$ and $V$ precisely act. In chapter 4.2 we have derived these operators in matrix form, in the subspace spanned by $\varphi_k^0$ and $\varphi_k^1$.

Lets consider a general state in this subspace of the form $\ket{\psi} = c_0 \ket{\varphi_k^0} + c_1 \ket{\varphi_k^1}$, which would look like the following:

$$
\begin{tikzpicture}
  \draw[thin,gray!40] (-2,-2) grid (2,2);
  \draw[<->] (-2,0)--(2,0) node[right]{$\varphi_k^0$};
  \draw[<->] (0,-2)--(0,2) node[above]{$\varphi_k^1$};
  \draw[line width=1,5pt,cyan,-stealth](0,0)--(1,1) node[anchor=south west]{$\boldsymbol{\psi}$};
\end{tikzpicture}
$$

Let us apply $S$ to $\psi$:

$$
\begin{pmatrix}
-1 & 0\\
0& 1
\end{pmatrix}
\begin{pmatrix}
c_0\\
c_1
\end{pmatrix}
=
\begin{pmatrix}
- c_0\\
c_1
\end{pmatrix}
\\
$$

A visual representation of what happens when S is applied to $\psi$:


$$
\begin{tikzpicture}
  \draw[thin,gray!40] (-2,-2) grid (2,2);
  \draw[<->] (-2,0)--(2,0) node[right]{$\varphi_k^0$};
  \draw[<->] (0,-2)--(0,2) node[above]{$\varphi_k^1$};
  \draw[line width=1,5pt,cyan,-stealth](0,0)--(1,1) node[anchor=south west]{$\boldsymbol{\psi}$};
  \draw[line width=1,5pt,green,-stealth](0,0)--(-1,1) node[anchor=south west]{$\boldsymbol{S\psi}$};
\end{tikzpicture}
$$

As is evident from the graph above, the $S$ operator reflects any state in the subspace around the $\varphi_k^1$ axis.
To see what effect $V$ has on a state in this subspace, we can simply apply $V$ to some state in the subspace. For efficiency reasons, we pick $c_0 = 1 \wedge c_1 = 0$.

\begin{align*}
V \ket{\psi} &= V \ket{\varphi_k^0}\\
&= \Vop \fiok\\
&= \sum_j \beta_j \ket{j} \otimes P_j \ket{\phi_k}
\end{align*}

From this result, we conclude that $V$ applies the specific Pauli operator, corresponding to the state of the system of qubits, to the $k_{th}$ eigenstate of the Hamiltonian. \textcite{poulin} define the unitary walk operator as:

$$
W = SVe^{i\pi}
$$

\subsection{Details of the Unitary Walk Operator}

\section{Quantum Phase Estimation}
%As $U$ is unitary, it's eigenvalues are unitary as well and can therefore be written in the form $e^{i\theta_k}$. The eigenstates are $\ket{\varphi_k^\pm} = (\ket{\varphi_k^0} \pm i \ket{\varphi_k^1})/\sqrt 2$.


Berry 2015 have a different convention, they refer to the V operator as the Select operator by

$$
\operatorname{select}(V)|j\rangle|\psi\rangle=|j\rangle V_{j}|\psi\rangle
$$

We know that B:
\[
B|0\rangle=\frac{1}{\sqrt{s}} \sum_{j=0}^{m-1} \sqrt{\beta_{j}}|j\rangle
\]
where we define \(s:=\sum_{j=0}^{m-1} \beta_{j} .\)

Definition for multi-qubit pauli operators



\subsection{Protocol}


When implementing the Spectrum by Quantum Walk algorithm, the following general protocol is to be followed:

\begin{enumerate}
  \item Normalize the Hamiltonian of the system to be simulated.
  \item Create two registers: an ancillary register and a system register of sizes $log(N)$ and $n$ respectively.
  \item Prepare an initial state of the form $ B \ket{0} \otimes \tilde{\ket{\phi_0}}$.
  \item Create an additional ancillary register of size $t$ and perform phase estimation of $W$ to obtain $\abs{\theta_k}$.
  \item Repeat this process to find the normalized energy, as $cos \theta_k = E_k$
  \item Multiply $E_k$ with $N$
\end{enumerate}


bl lal 


% General notes: Normally, the phase estimation would be performed by Hamiltonian simulation. That introduces two difficulties: first, there is error introduced by the Hamiltonian simu- lation that needs to be taken into account in bounding the overall error, and second, there can be ambiguities in the phase that require simulation of the Hamiltonian over very short times to eliminate.
% These problems can be eliminated if one were to
% use Hamiltonian simulation via a quantum walk, as in
% Refs. [53, 54]. There, steps of a quantum walk can be
% performed exactly, which have eigenvalues related to the
% eigenvalues of the Hamiltonian. Specifically, the eigen-
% ±iarcsin(E /λ)
% values are of the form ±e k . Instead of using
% Hamiltonian simulation, it is possible to simply perform phase estimation on the steps of that quantum walk, and invert the function to find the eigenvalues of the Hamil- tonian. That eliminates any error due to Hamiltonian simulation. Moreover, the possible range of eigenvalues of the Hamiltonian is automatically limited, which elim- inates the problem with ambiguities. from similar method

\newpage





\printbibliography

\end{document}
